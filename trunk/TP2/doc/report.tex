\documentclass{article}

\usepackage{listings}
\usepackage{color}
\usepackage{graphicx}
\usepackage{float}
\usepackage{amsmath}
\usepackage{subfig}
\usepackage{cite}
\usepackage{url}

\begin{document}

\title{Image Analysis - TP2 - Local Filtering and Histograms}

\author{Jander Nascimento, 
\and Raquel Oliveira}

\maketitle

\section{Filtering}
	
	\subsection{Binomial}

		Binomial filter uses Pascal's triangle to create a filter. This filter has the blurry effect with some differences in preserving objects in the image.

One example of a 3x3 Binomial filter can be seen in the Figure \ref{3x3binomial}.

The values for this kernel can be obtained from binomial coefficient definition, presented in Equation \ref{equa:bc}.

\begin{equation}
\frac{n!}{(n-k)!k!}={n \choose k}
\label{equa:bc}
\end{equation}

\begin{figure}[H]
  \begin{center}
  \begin{tabular}{ | c | c | c | }
    \hline
    1 & 2 & 1 \\ \hline

    2 & 4 & 2 \\ \hline

    1 & 2 & 1 \\
    \hline
  \end{tabular}
  \end{center}
  \caption{3 x 3 Binomial filter\label{3x3binomial}}\end{figure}

		To apply this kernel to the image, it is necessary to perform the convolution in the original image with this kernel, the convolution is not part of this present report.

		This kernel is used to remove the noise in image, but it do not preserves the edges of the image. 
		
	\subsection{Median}

		With a purpose of noise reduction, in certain situations it can preserve the edges while reduce the noise of the image.

		For this reason this kind of filter is regularly used as a pre-treatment for edge detection.

		The Median filtering differs from others filters in the way it is applied to the image. While other filters are defined by kernels as the aproximation of its derivation, the median filter depends on each sector of the image analyzed to calculate the new pixel intensities.

		Median kernel copies a certain sector of the image analyzed and change the value of the central pixel in the original image. 

		The central pixel in the original image receives the median value of the kernel. It is possible to do it by simply sorting the content of the kernel and assign the value in the middle (central position of the kernel) to the original image. 

	\subsection{Binomial {\it versus} Median}

		\begin{figure}[H]
		  \centering
		  \subfloat[Binomial]{\label{fig:binomial}\includegraphics[width=0.4\textwidth]{../image/church_binomial}}                
		  \subfloat[Median]{\label{fig:median}\includegraphics[width=0.4\textwidth]{../image/church_median}}
		  \caption{Removing noise}
		  \label{fig:removingnoise}
		\end{figure}		

	In the Figure \ref{fig:removingnoise} we applied Binomial and Median filter separately. 
	Median preserves the edges better than the Binomial, although the Binomial preserves some central details in the image that the Binomial do not. 
	This behavior can be seen in the window at the chapel, in the Binomial is possible to see the borders of the window more clearly, while in the Median this detail is not displayed at all. 

\section{Histogram}

	The histogram is an important tool when dealing with image analysis. It shows the color distribution of a certain image regarding to its color intensity. The 		histogram is represented in a Cartesian, where x axis is the color intensity and y represents the number of pixels that have such intensity.

	In this practical work, we wrote a function in c that computes the intensity histogram. The function has the following steps:
	\begin{itemize}
  		\item take from the main argument the name of the file that will be analyzed.
  		\item create a matrix (with the dimensions of the image) to store the pixels' intensity of the image
  		\item go through such matrix and count the number of pixels in each intensity at the intensity of the color (from 0 to 255)
  		\item store that in an array of 256 positions.
  		\item display such array on the standard output device (screen)
	\end{itemize}

	\begin{figure}[H]
	\centering
	\subfloat[Image]{\label{fig:image}\includegraphics[width=0.4\textwidth]{../image/amelie}}                
	\subfloat[Histogram]{\label{fig:histogram}\includegraphics[width=0.4\textwidth]{../image/histogram}}
	\caption{Histrogram of an image}
	\label{fig:equalizing}
	\end{figure}

	\subsection{Stretching}

	The histogram stretching is a technique by which the color histogram is used to evaluate and possibly change the color intensity range. This enhances the 		detail level of some images that might appear too dark or too bright. This is done by spreading the pixels to use the entire color intensity.
	
	In this practical work, we wrote a function in c that transforms an image with the histogram stretching. The function has the following steps:
	\begin{itemize}
  		\item take from the main argument the name of the file that will be analyzed.
  		\item create a matrix (with the dimensions of the image) to store the pixels' intensity of the image
  		\item in the matrix, find the current range minimum of the image({\it $current_{min}$}), which is the minimum intensity of the intensity of the color 			that is used in the image
  		\item find also the current range maximum of the image({\it $current_{max}$}), which is the maximum intensity of the intensity of the color that is 			used in the image
  		\item for each pixel of the image, calculate its new intensity based on the stretching formula:
		\begin{equation}
			y[n]=new_{min}+\frac{new_{max}-new_{min}}{current_{max}-current_{min}}*(x[n]-current_{min})
			\label{eq:stretching}
		\end{equation}
	
		where: 
			\begin{itemize}
	  			\item {\it n} is the pixel		
	  			\item {\it $new_{min}$} is 0;
	  			\item {\it $new_{max}$} is 255; 
	  			\item {\it x[n]} is the current intensity of the pixel.
			\end{itemize}
	\end{itemize}

	\begin{figure}[H]
	\centering
	\subfloat[Original Image]{\label{fig:original}\includegraphics[width=0.4\textwidth]{../image/amelie}}                
	\subfloat[Stretched Image]{\label{fig:stretched}\includegraphics[width=0.4\textwidth]{../image/amelie_stretched}}
	\subfloat[Stretched Histogram]{\label{fig:histogram_s}\includegraphics[width=0.4\textwidth]{../image/histogram_stretching}}
	\caption{Stretching Transformation}
	\label{fig:stretching}
	\end{figure}

	\subsection{Equalization}

	Histogram equalization is a method in image processing of contrast adjustment using the image's histogram. This method usually increases the global contrast 		of many images, especially when the usable data of the image is represented by close contrast values. Through this adjustment, the intensities can be better 		distributed on the histogram. This allows for areas of lower local contrast to gain a higher contrast. Histogram equalization accomplishes this by 		effectively spreading out the most frequent intensity values.

	In this practical work, we wrote a function in c that transforms an image with the histogram equalization. The function has the following steps:
	\begin{itemize}
  		\item compute the histogram of the image (following the steps of the question before)
  		\item create an array to store the cumulative distribution function of the histogram, which is the values of the histogram in a cumulative way.
  		\item in such array, find the minimum value at all the intensity of the color ($cdf_{min}$)
  		\item for each pixel of the image, calculate its new intensity based on the equalization formula:
			\begin{equation}
				h[v]= round \left( \frac{cdf(v)-cdf_{min}}{(M * N) - cdf_{min}} * (L-1) \right) 
			\label{eq:equalization}
			\end{equation}
			
			where:		
			\begin{itemize}
	  			\item {\it v} is the current intensity of the pixel		
		  		\item {\it cdf(v)} is the value of such intensity in the array that stores the accumulated values of the histogram
		  		\item {\it M} is the height of the image		
		  		\item {\it N} is the width of the image
		  		\item {\it L} is the number of color levels used (in most cases, 256)
			\end{itemize}
	\end{itemize}

	\begin{figure}[H]
	\centering
	\subfloat[Original Image]{\label{fig:original2}\includegraphics[width=0.4\textwidth]{../image/amelie}}                
	\subfloat[Equalized Image]{\label{fig:equalized}\includegraphics[width=0.4\textwidth]{../image/amelie_equalized}}
	\subfloat[Equalized Histogram]{\label{fig:histogram_e}\includegraphics[width=0.4\textwidth]{../image/histogram_equalized}}
	\caption{Equalization Transformation}
	\label{fig:equalizing}
	\end{figure}


	\subsection{Where to find the source code?}

	The code is hosted at \url{https://jfimageanalysis.googlecode.com/svn/trunk/TP2/}. 
	To download the code execute this command:
	
	svn checkout https://jfimageanalysis.googlecode.com/svn/trunk/TP2/
	\\Remark: you {\bf must} have installed subversion. 
	
	\subsection{How to run?}
	First compile the source (using {\it make}). The application name is {\it imagetransform} 
	
	Here are some examples:
	
	Apply binomial filter to an image:
	\begin{lstlisting}	
		imagetransform -f binomial -i image_in.pgm > image_out.pgm
	\end{lstlisting}
	You still can combine this command to specify the size of the kernel (-s) or the number of times that the filter should be applied (-n)	
	
	Apply median filter to an image:
	\begin{lstlisting}	
		imagetransform -f median -i image_in.pgm > image_out.pgm
	\end{lstlisting}	

	And of course you can always type {\it imagetransform --help} to check for more options. 

\end{document}


