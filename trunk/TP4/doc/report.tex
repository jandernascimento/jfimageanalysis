\documentclass{article}

\usepackage{listings}
\usepackage{color}
\usepackage{graphicx}
\usepackage{float}
\usepackage{amsmath}
\usepackage{subfig}
\usepackage{cite}
\usepackage{url}
\usepackage{amsmath}

\begin{document}

\title{Image Analysis - TP4 - K Means}

\author{Jander Nascimento, 
\and Raquel Oliveira}

\maketitle

\section{Initial concept}

The \emph{k-means} split the image in k number of sets based in some parameters. Those parameters are heuritics to split the set, one common way to splits the sets is in a 3d dimensional space, in which the domain of the space is [0,255] and each axle is the color space RGB. 

\section{Initial values influent}

As the \emph{k} represents the number of sets, this determines the distinct number of regions that exists in the picture, so as more \emph{k} are chosen, more detailed the image will be.  

\section{Number of regions}
\section{Other influences}

\end{document}


