\documentclass{article}

\usepackage{listings}
\usepackage{color}
\usepackage{graphicx}
\usepackage{float}
\usepackage{amsmath}
\usepackage{subfig}
\usepackage{cite}
\usepackage{url}
\usepackage{amsmath}

\newcounter{qcounter}

\begin{document}

\title{Image Analysis - TP5 - Lucas Kanade Tracker}

\author{Jander Nascimento, 
\and Raquel Oliveira}

\maketitle

\section{Initial concept}


\section{Interpolation}


\section{Analysis of a static image}


\section{Tracking an object}


\section{Drawbacks}

Lucas Kanade has shown to be very efficient in some cases, now let's take a look in the situation where this algorithm does not work that well.

In large motions (at a higher speed) the tracker is not able to detect the changing region of the image with the correct direction vector.

If the image changes drastically its illumination, this can produce a wrong image evaluation.

\section{How to run?}

	Steps to compile the application:
	
	\begin{itemize}
		\item svn checkout https://jfimageanalysis.googlecode.com/svn/trunk/TP4/ \#download source code
		\item make \#compiles the code
	\end{itemize}

	As an input image only {\bf ppm plaintext/ansii} files are accepted (P3). 

	Examples of usage:

	\begin{itemize}
		\item Create 3 clusters:
		\subitem ./showregion -i image\_in.ppm -g 3 $>$ image\_out.ppm
	\end{itemize}

	You can always type {\it ./tp5 --help} to check for more options. 

\end{document}


